%%%%%%%%%%%%%%%%%%%%%%%%%%%%%%%%%%%%%%%%%%%%%%%%%%%%%%%%%%%%%%%%%%
% Arquivo LaTeX geral para a classe Article
% Uso específico: exercícios sobre algoritmos
%
% Abrantes Araújo Silva Filho
% abrantesasf@gmail.com
% 2016-10-12


%%%%%%%%%%%%%%%%%%%%%%%%%%%%%%%%%%%%%%%%%%%%%%%%%%%%%%%%%%%%%%%%%%
%%% Configura tipo de documento e load de packages:
\RequirePackage{ifpdf}
\ifpdf
  \documentclass[pdftex,a4paper,12pt,brazil]{article} % Se tem draft é rascunho
  %\usepackage{ae}
  \usepackage[pdftex]{geometry}
  \usepackage[pdftex]{graphicx}
  \geometry{a4paper,left=2cm,right=2cm,top=2cm,bottom=2cm}
  \usepackage{setspace}
  \usepackage[T1]{fontenc}
  \usepackage[utf8]{inputenc}
  \usepackage[brazil]{babel}
  \usepackage[brazil]{varioref}
  \usepackage[pdftex,pdfpagemode=UseOutlines,bookmarks=true,%
   bookmarksopen=true,bookmarksopenlevel=5,bookmarksnumbered=true,%
   pdfstartview=FitH,hyperfootnotes=true]{hyperref}
   \hypersetup{pdfinfo={
   Author={Abrantes Ara\'{u}jo Silva Filho},
   Title={Respostas do cap\'{i}tulo 1 do livro: Introduction to Algorithms},
   Creator={pdfLaTeX},
   Producer={pdfTeX},
   CreationDate={},
   ModDate={},
   Subject={Estudo sobre algoritmos},
   Keywords={algoritmos, algorithms, cormen, homework, exerc\'{i}cios, respostas, solutions},
   }}
  %\usepackage{thumbpdf}
  \hypersetup{colorlinks,%
    debug=false,%
    linkcolor=blue,%
    citecolor=blue,%
    urlcolor=blue}
  \usepackage{cleveref}
  \mathchardef\period=\mathcode`.
\else
  \documentclass[a4paper,12pt]{article}
  \usepackage[utf8]{inputenc}
  \usepackage[T1]{fontenc}
  \usepackage[brazil]{babel}
  \usepackage[dvips]{geometry}
  \usepackage[dvips]{graphicx}
  \geometry{a4paper,left=2cm,right=2cm,top=2cm,bottom=2cm}
  \usepackage{setspace}
  \usepackage{varioref}
  \usepackage{hyperref}
  \usepackage{cleveref}
\fi


%%%%%%%%%%%%%%%%%%%%%%%%%%%%%%%%%%%%%%%%%%%%%%%
%%% Configura lingua portuguesa:
%\usepackage[brazil]{babel}
%\usepackage[utf8]{inputenc}
%\usepackage[T1]{fontenc}


%%%%%%%%%%%%%%%%%%%%%%%%%%%%%%%%%%%%%%%%%%%%%%%
%%% Configura símbolos e bibliotecas matemáticas:
\usepackage{amsmath}
\usepackage{amssymb}
\usepackage{latexsym}
%\usepackage{physics}
\usepackage{siunitx}
\sisetup{group-separator = {.}}
\sisetup{group-digits = {false}}
\sisetup{output-decimal-marker = {,}}


%%%%%%%%%%%%%%%%%%%%%%%%%%%%%%%%%%%%%%%%%%%%%%%
%%% Configura fontes e outros símbolos
\usepackage{wasysym}
\usepackage{pifont}
\usepackage{marvosym}


%%%%%%%%%%%%%%%%%%%%%%%%%%%%%%%%%%%%%%%%%%%%%%%
%%% Ativa pacote ifthen, necessário para alguns comandos
\usepackage{ifthen}


%%%%%%%%%%%%%%%%%%%%%%%%%%%%%%%%%%%%%%%%%%%%%%%
%%% Ativa suporte a cores:
\usepackage{color}
\usepackage[dvipsnames]{xcolor}
\usepackage{xparse}


%%%%%%%%%%%%%%%%%%%%%%%%%%%%%%%%%%%%%%%%%%%%%%%
%%% Ativa figuras e tabelas
\usepackage{float}
\usepackage{wrapfig}


%%%%%%%%%%%%%%%%%%%%%%%%%%%%%%%%%%%%%%%%%%%%%%%
%%% Ativa suporte ao TikZ Code
\usepackage{tikz}
\usetikzlibrary{positioning,shapes,shadows}


%%%%%%%%%%%%%%%%%%%%%%%%%%%%%%%%%%%%%%%%%%%%%%%
%%% Ativa pacote para tabelas longas e em landscape
\usepackage{array,longtable}
\usepackage{lscape}
\usepackage{array}
\usepackage{colortbl}
\newcolumntype{M}[1]{>{\centering\arraybackslash}m{#1}}
\newcolumntype{L}[1]{>{\arraybackslash}m{#1}}
\newcolumntype{N}{@{}m{0pt}@{}}


%%%%%%%%%%%%%%%%%%%%%%%%%%%%%%%%%%%%%%%%%%%%%%%
%%% Ativa pacote para URLs, e-mails e pathmanes:
\usepackage{url}


%%%%%%%%%%%%%%%%%%%%%%%%%%%%%%%%%%%%%%%%%%%%%%%
%%% Commando para ``italizar´´ palavras em inglês (e outras línguas!)
\newcommand{\ingles}[1]{\textit{#1}}


%%%%%%%%%%%%%%%%%%%%%%%%%%%%%%%%%%%%%%%%%%%%%%%
%%% Commando para colocar o espaço correto entre um número e sua unidade
\newcommand{\unidade}[2]{\ensuremath{#1\,\mathrm{#2}}}
\newcommand{\unidado}[2]{{#1}\,{#2}}


%%%%%%%%%%%%%%%%%%%%%%%%%%%%%%%%%%%%%%%%%%%%%%%%%%%%%%%%%%%%
%% produz ordinal masculino ou feminino dependendo do segundo
%% argumento.  Por exemplo:
%% \ordinal{1}{a} Semana
%% \ordinal{1}{o} Encontro
\newcommand{\ordinal}[2]{%
#1%
\ifthenelse{\equal{a}{#2}}%
{\textordfeminine}%
{\textordmasculine}}


%%%%%%%%%%%%%%%%%%%%%%%%%%%%%%%%%%%%%%%%%%%%%%%
%%% Ativa suporte a sublinhado:
% A opção normalem indica que ênfase será dada por itálico
% e não por sublinhado.
\usepackage[normalem]{ulem}
% O código a seguir define mais comandos para o pacote ulem, e foi retirado
% do "LaTeX demo: exemplos com LaTeXe", de Klauss Steding Jessen:
\def\dotuline{\bgroup
  \ifdim\ULdepth=\maxdimen  % Set depth based on font, if not set already
  \settodepth\ULdepth{(j}\advance\ULdepth.4pt\fi
  \markoverwith{\begingroup
  \advance\ULdepth0.08ex
  \lower\ULdepth\hbox{\kern.15em .\kern.1em}%
  \endgroup}\ULon}
\def\dashuline{\bgroup
  \ifdim\ULdepth=\maxdimen  % Set depth based on font, if not set already
  \settodepth\ULdepth{(j}\advance\ULdepth.4pt\fi
  \markoverwith{\kern.15em
  \vtop{\kern\ULdepth \hrule width .3em}%
  \kern.15em}\ULon}


%%%%%%%%%%%%%%%%%%%%%%%%%%%%%%%%%%%%%%%%%%%%%%%
%%% Ativa pacote para indentação da primeira linha de parágrafos
%\usepackage{indentfirst}


%%%%%%%%%%%%%%%%%%%%%%%%%%%%%%%%%%%%%%%%%%%%%%%
%%% Ativa pacote enumerate, extensão ao environment enumerate:
\usepackage{enumerate}


%%%%%%%%%%%%%%%%%%%%%%%%%%%%%%%%%%%%%%%%%%%%%%%
%%% environment ``Description'', similar ao environment
%%% ``description'', mas com maior controle sobre a tabulação das
%%% entradas e de suas descrições.
%%% Adaptado de um exemplo do LaTeX Companion, pg. 64.
\newlength{\myentrylen}
\newenvironment{Description}[1]%
{\list{}
  {\settowidth{\labelwidth}{\textbf{#1}}%
    \leftmargin\labelwidth\advance\leftmargin\labelsep%
    \renewcommand{\makelabel}[1]{%
      \settowidth{\myentrylen}{\textbf{##1}}%
      \ifthenelse{\lengthtest{\myentrylen > \labelwidth}}%
      {\parbox[b]{\labelwidth}%
        {\makebox[0pt][l]{\textbf{##1}}\\\mbox{}}}
      {\textbf{##1}}%
      \hfill\relax%
      }
}}
{\endlist}


%%%%%%%%%%%%%%%%%%%%%%%%%%%%%%%%%%%%%%%%%%%%%%%
%%% Comando para epígrafe em capítulos/seções (não confundir com a epígrafe geral
% da tese, definida em página isolada nos elementos pré-textuais.
\newcommand{\epigrafe}[2]{
   \vspace{-6ex}%
     {\footnotesize%
     \begin{flushright}%
     \begin{minipage}{.6\textwidth}%
     #1
     \end{minipage}\\
     \textit{#2}%
     \end{flushright}}%
   \vspace{-3ex}}


%%%%%%%%%%%%%%%%%%%%%%%%%%%%%%%%%%%%%%%%%%%%%%%
%%% Ativa pacote para controle de cabeçalhos e rodapés e configura;
% Ativa pacote:
\usepackage{fancyhdr}
% Configura estilo padrão das páginas
\pagestyle{headings}


%%%%%%%%%%%%%%%%%%%%%%%%%%%%%%%%%%%%%%%%%%%%%%%
%%% Ativa pacote para formatar os captions:
\usepackage[normal,bf]{caption}
%\captionsetup[table]{font=small,skip=0pt}
\captionsetup[figure]{skip=0pt}


%%%%%%%%%%%%%%%%%%%%%%%%%%%%%%%%%%%%%%%%%%%%%%%
%%% Ativa o MakeIndex para fazer índices remissivos:
\usepackage{makeidx}
%\makeindex


%%%%%%%%%%%%%%%%%%%%%%%%%%%%%%%%%%%%%%%%%%%%%%%
%%% Ativa pacote para fazer glossário, conforme
% "LaTeX demo: exemplos com LaTeXe", de Klauss Steding Jessen.
\usepackage{makeglo}
%\makeglossary


%%%%%%%%%%%%%%%%%%%%%%%%%%%%%%%%%%%%%%%%%%%%%%%
%%% Ativa o pacote havard de referências bibliográficas e
% define um novo comando para colocar as citações em slanted:
% As opções do pacote determinam como as citações aparecem no texto.
% As seguinte opções existem:
%      default: lista a primeira completa e as subsequentes abreviadas;
%	  full: lista todas as citações completas;
%         abbr: lista todas as citações abreviadas.
% A qualquer momento o modo de citaçõa pode ser alterado com o uso do
% comando: \citationmode{}, cujo argumento é uma das opções da lista anterior.
\usepackage[default]{harvard}
% Cria comando para colocar as citações em slanted:
\newcommand{\refbib}[1]{\textsl{#1}}
% O seguinte comando configura como as referências bibliográficas serão
% formatadas. O estilo agsm é o padrão do pacote havard. Ver manual de instrução
% do pacote para maiores informações.
\bibliographystyle{agsm}
% Configura como as citações das referências apareceram no texto. O estilo agsm
% é o padrão do pacote havard. Ver manual de instrução do pacote para maiores
% informações.
\citationstyle{agsm}


%%%%%%%%%%%%%%%%%%%%%%%%%%%%%%%%%%%%%%%%%%%%%%%
%%% Comandos específicos para este documento


%%%%%%%%%%%%%%%%%%%%%%%%%%%%%%%%%%%%%%%%%%%%%%%
%%% Determina forma de hifenização de palavras quando a hifenização
%%% padrão não estiver correta
%\hyphenation{ne-nhu-ma}
\babelhyphenation[brazil]{ne-nhu-ma}


%%%%%%%%%%%%%%%%%%%%%%%%%%%%%%%%%%%%%%%%%%%%%%%%%%%%%%%%%%%%%%%%%%%%%%%%%%%%%%%%%%%%%%%%%%%%%%
%%%%%%%%%%%%%%%%%%%%%%%%%%%%%%%%%%%%%%%%%%%%%%%%%%%%%%%%%%%%%%%%%%%%%%%%%%%%%%%%%%%%%%%%%%%%%%
%%%%%%%%%%%%%%%%%%%%%%%%%%%%%%%%%%%%%%%%%%%%%%%%%%%%%%%%%%%%%%%%%%%%%%%%%%%%%%%%%%%%%%%%%%%%%%
%%%%%%%%%%%%%%%%%%%%%%%%%%%%%%%%%%%%%%%%%%%%%%%%%%%%%%%%%%%%%%%%%%%%%%%%%%%%%%%%%%%%%%%%%%%%%%
%%%%%%%%%%%%%%%%%%%%%%%%%%%%%%%% COMEÇA DOCUMENTO %%%%%%%%%%%%%%%%%%%%%%%%%%%%%%%%%%%%%%%%%%%%
%%%%%%%%%%%%%%%%%%%%%%%%%%%%%%%%%%%%%%%%%%%%%%%%%%%%%%%%%%%%%%%%%%%%%%%%%%%%%%%%%%%%%%%%%%%%%%
%%%%%%%%%%%%%%%%%%%%%%%%%%%%%%%%%%%%%%%%%%%%%%%%%%%%%%%%%%%%%%%%%%%%%%%%%%%%%%%%%%%%%%%%%%%%%%
%%%%%%%%%%%%%%%%%%%%%%%%%%%%%%%%%%%%%%%%%%%%%%%%%%%%%%%%%%%%%%%%%%%%%%%%%%%%%%%%%%%%%%%%%%%%%%
%%%%%%%%%%%%%%%%%%%%%%%%%%%%%%%%%%%%%%%%%%%%%%%%%%%%%%%%%%%%%%%%%%%%%%%%%%%%%%%%%%%%%%%%%%%%%%
\begin{document}
\title{Respostas do capítulo 1 do livro:\\
  \emph{Introduction to Algorithms},\\
  de Cormen, Thomas H.\ et al.\ (\ordinal{3}{a} ed., 2009)}
\author{Abrantes Araújo Silva Filho}
\date{2018-01}
\maketitle
\tableofcontents
%\newpage


%%%%%%%%%%%%%%%%%%%%%%%%%%%%%%%%%%%%%%%%%%%%%%%%%%%%%%%%%%%%%%%%%%%%%%%%%%%%%%%%%%%%%%%%%%%%%%
%%%%%%%%%%%%%%%%%%%%%%%%%%%%%%%%%%%%%%%%%%%%%%%%%%%%%%%%%%%%%%%%%%%%%%%%%%%%%%%%%%%%%%%%%%%%%%
%%%%%%%%%%%%%%%%%%%%%%%%%%%%%%%%%%%%%%%%%%%%%%%%%%%%%%%%%%%%%%%%%%%%%%%%%%%%%%%%%%%%%%%%%%%%%%
%%%%%%%%%%%%%%%%%%%%%%%%%%%%%%%%%%%%%%%%%%%%%%%%%%%%%%%%%%%%%%%%%%%%%%%%%%%%%%%%%%%%%%%%%%%%%%
%%%%%%%%%%%%%%%%%%%%%%%%%%%%%%%%%%%%%%%%%%%%%%%%%%%%%%%%%%%%%%%%%%%%%%%%%%%%%%%%%%%%%%%%%%%%%%
\section{O que é este documento?} 
\label{o_que_e}
%\thispagestyle{plain}

Este documento contém as minhas respostas aos exercícios e problemas do capítulo 1 do livro
\emph{Introduction to Algorithms}, de Cormen, Thomas H.\ et al.\ (\ordinal{3}{a} ed., de 2009),
que utilizei na disciplina de Algoritmos durante minha gradução em Ciência da Computação.

ATENÇÃO: não garanto que tudo aqui está correto, pelo contrário, algumas respostas expressam
minha visão particular e podem estar em desacordo com
a ``resposta padrão'' dos autores do livro ou do professor da disciplina de Algoritmos. Também
não garanto que todos os exercícios e problemas do capítulo estarão resolvidos aqui.

De qualquer modo, se você quiser
utilizar este documento como base para seu próprio estudo, tenha em mente o seguinte:

ESTE DOCUMENTO É FORNECIDO ``NO ESTADO EM QUE SE ENCONTRA'', SEM GARANTIAS DE QUALQUER
NATUREZA, EXPRESSAS OU IMPLÍCITAS. EM NENHU\-MA HIPÓTESE O AUTOR PODERÁ SER RESPONSABILIZADO POR
QUALQUER RECLAMAÇÃO, DANOS OU OUTROS PROBLEMAS DECORRENTES DO USO DESTE CONTEÚDO.

Este documento (em formato PDF) e o original em \LaTeX\ estão disponíveis no seguinte
repositório GitHub: \url{https://github.com/abrantesasf/algoritmos}


%%%%%%%%%%%%%%%%%%%%%%%%%%%%%%%%%%%%%%%%%%%%%%%%%%%%%%%%%%%%%%%%%%%%%%%%%%%%%%%%%%%%%%%%%%%%%%
%%%%%%%%%%%%%%%%%%%%%%%%%%%%%%%%%%%%%%%%%%%%%%%%%%%%%%%%%%%%%%%%%%%%%%%%%%%%%%%%%%%%%%%%%%%%%%
%%%%%%%%%%%%%%%%%%%%%%%%%%%%%%%%%%%%%%%%%%%%%%%%%%%%%%%%%%%%%%%%%%%%%%%%%%%%%%%%%%%%%%%%%%%%%%
%%%%%%%%%%%%%%%%%%%%%%%%%%%%%%%%%%%%%%%%%%%%%%%%%%%%%%%%%%%%%%%%%%%%%%%%%%%%%%%%%%%%%%%%%%%%%%
%%%%%%%%%%%%%%%%%%%%%%%%%%%%%%%%%%%%%%%%%%%%%%%%%%%%%%%%%%%%%%%%%%%%%%%%%%%%%%%%%%%%%%%%%%%%%%
\section{Exercícios} 
\label{exercicios}
%\thispagestyle{plain}


%%%%%%%%%%%%%%%%%%%%%%%%%%%%%%%%%%%%%%%%%%%%%%%%%%%%%%%%%%%%%%%%%%%%%%%%%%%%%%%%%%%%%%%%%%%%%%
%%%%%%%%%%%%%%%%%%%%%%%%%%%%%%%%%%%%%%%%%%%%%%%%%%%%%%%%%%%%%%%%%%%%%%%%%%%%%%%%%%%%%%%%%%%%%%
\subsection{Grupo 1.1:}
\label{grupo_1_1}

\paragraph{Exercício 1.1-1}

Um exemplo real da necessidade de algoritmos que necessitam de ordenação
(sorting) é a ordenação alfabética de uma lista de palavras para a criação de um dicionário. Ou
a criação de um ranking nacional com as notas finais de todos os estudantes brasileiros que
participaram do Enem de 2017.

\paragraph{Exercício 1.1-2}

Além da velocidade (tempo de execução), outras medidas de eficiência de
um algoritmo podem incluir:

\begin{itemize}
\item Uso de memória
\item Uso de IO de disco
\item Uso de banda de rede
\item Uso de random bits
\end{itemize}

\paragraph{Exercício 1.1-3}

Uma estrutura de dados que já vi antes é o \emph{vetor}, que é um array
unidimensional de $n$ números. É bom para cálculos, mas só armazena dados de um mesmo tipo.

\paragraph{Exercício 1.1-4}

O problema da menor distância entre dois pontos em um mapa e o problema
do caixeiro viajante são semelhantes no sentido de que ambos os problemas tratam de
distâncias a serem percorridas, mas são muito diferentes quanto a complexidade da tarefa
computacional. Encontrar somente a menor distância entre dois pontos pode ser resolvido
algoritmicamente com eficiência, ao passo que o problema do caixeiro viajante é um problema
NP-completo, ou seja, ainda não existe uma solução ótima eficiente para sua resolução.

\paragraph{Exercício 1.1-5}

Um problema na qual apenas a melhor solução é aceitável seria, por exemplo,
a identificação de um possível objeto voador que vem em nossa direção como um ICBM atômico
ou outro objeto qualquer: identificar erroneamente um objeto como ICBM atômico pode iniciar
uma guerra nuclear, portanto o algoritmo de identificação deve ser ótimo.

Um problema no qual uma solução próxima da melhor é aceitável seria, por exemplo,
determinar a raiz quadrada de um número com 30 casas decimais.

%%%%%%%%%%%%%%%%%%%%%%%%%%%%%%%%%%%%%%%%%%%%%%%%%%%%%%%%%%%%%%%%%%%%%%%%%%%%%%%%%%%%%%%%%%%%%%
\subsection{Grupo 1.2:}
\label{grupo_1_2}

\paragraph{Exercício 1.2-1}

Um exemplo de aplicação que requer o uso de algorítmos no nível da aplicação é um serviço
web que determina como viajar de uma localização à outra: seriam necessários algoritmos
para determinar o caminho mais curto (ou outro tipo de caminho especificado pelo usuário,
tais como rotas sem pedágio), a renderização de mapas e a interpolação de endereços.

\paragraph{Exercício 1.2-2}

Para algoritmos de insertion sort que rodam em $8n^2$ passos e algoritmos de merge sort
que rodam em $64n \lg n$ passos, o insertion sort será mais rápido do que o merge sort quando:

\begin{equation}
  \begin{split}
  8n^2 & < 64n \lg n =\\
  n^2 & < 8n \lg n =\\
  n &< 8 \lg n
  \end{split}
\end{equation}

Assim, sempre que $n < 8 \lg n$, onde $\lg$ é o logarítmo de base 2, o insertion sort será
mais rápido do que o merge sort. Fazendo uma rápida tabela, podemos constatar que isso somente
ocorre quando $2 \le n \le 43$. Para qualquer $n > 43$, o merge sort será mais rápido do que
o insertion sort, nos tempos de execução informados pelo problema.

\paragraph{1.2-3}

O menor valor de $n$ para que um algoritmo cujo tempo de execução é de $100n^2$ rode mais
rápido do que outro algoritmo com tempo de execução $2^n$ na mesma máquina,

\begin{equation}
  100n^2 < 2^n
\end{equation}

\noindent também pode ser obtido com uma planilha, e é de $n = 15$.

%%%%%%%%%%%%%%%%%%%%%%%%%%%%%%%%%%%%%%%%%%%%%%%%%%%%%%%%%%%%%%%%%%%%%%%%%%%%%%%%%%%%%%%%%%%%%%
%%%%%%%%%%%%%%%%%%%%%%%%%%%%%%%%%%%%%%%%%%%%%%%%%%%%%%%%%%%%%%%%%%%%%%%%%%%%%%%%%%%%%%%%%%%%%%
%%%%%%%%%%%%%%%%%%%%%%%%%%%%%%%%%%%%%%%%%%%%%%%%%%%%%%%%%%%%%%%%%%%%%%%%%%%%%%%%%%%%%%%%%%%%%%
%%%%%%%%%%%%%%%%%%%%%%%%%%%%%%%%%%%%%%%%%%%%%%%%%%%%%%%%%%%%%%%%%%%%%%%%%%%%%%%%%%%%%%%%%%%%%%
%%%%%%%%%%%%%%%%%%%%%%%%%%%%%%%%%%%%%%%%%%%%%%%%%%%%%%%%%%%%%%%%%%%%%%%%%%%%%%%%%%%%%%%%%%%%%%
%\newpage
\section{Problemas}
\label{problemas}
%\thispagestyle{plain}

aqui

%%%%%%%%%%%%%%%%%%%%%%%%%%%%%%%%%%%%%%%%%%%%%%%%%%%%%%%%%%%%%%%%%%%%%%%%%%%%%%%%%%%%%%%%%%%%%%
\subsection{Por que comprar a HP-12C, a ``calculadora que não morre''?}
\label{por-que-comprar}

çlkasdjçalksdjfçalskdfj


%%%%%%%%%%%%%%%%%%%%%%%%%%%%%%%%%%%%%%%%%%%%%%%
%%% Produz as referências bibliográficas
%% Configura título das referências bibliográficas:
%\newpage
%\renewcommand{\refname}{Referências bibliográficas}
%% Ativa arquivo com as referências bibliográficas:
%\bibliography{evandro}
%% Adiciona entrada na toc:
%\addcontentsline{toc}{section}{Referências bibliográficas}
%% Estilo da página
%\thispagestyle{headings}
%\index{referencias bibliograficas@Referências bibliográficas}


% Termina o documento
\end{document}             
