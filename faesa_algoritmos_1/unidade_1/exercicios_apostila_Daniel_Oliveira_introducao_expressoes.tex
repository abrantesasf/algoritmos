%%%%%%%%%%%%%%%%%%%%%%%%%%%%%%%%%%%%%%%%%%%%%%%%%%%%%%%%%%%%%%%%%%
% Arquivo LaTeX geral para a classe Article
% Uso específico: exercícios sobre expressões
%
% Abrantes Araújo Silva Filho
% abrantesasf@gmail.com
% 2018-02-25


%%%%%%%%%%%%%%%%%%%%%%%%%%%%%%%%%%%%%%%%%%%%%%%%%%%%%%%%%%%%%%%%%%
%%% Configura tipo de documento e load de packages:
\RequirePackage{ifpdf}
\ifpdf
  \documentclass[pdftex,a4paper,12pt,brazil]{article} % Se tem draft é rascunho
  %\usepackage{ae}
  \usepackage[pdftex]{geometry}
  \geometry{a4paper,left=2cm,right=2cm,top=2cm,bottom=2cm}
  \usepackage[pdftex]{graphicx}
  \usepackage{setspace}
  \usepackage[T1]{fontenc}
  \usepackage[utf8]{inputenc}
  \usepackage[brazil]{babel}
  \usepackage[brazil]{varioref}
  \usepackage[pdftex,pdfpagemode=UseOutlines,bookmarks=true,%
   bookmarksopen=true,bookmarksopenlevel=5,bookmarksnumbered=true,%
   pdfstartview=FitH,hyperfootnotes=true]{hyperref}
   \hypersetup{pdfinfo={
   Author={Abrantes Ara\'{u}jo Silva Filho},
   Title={Respostas de exerc\'{i}cios selecionados da apostila: Conceitos de Algoritmos (Daniel Barbosa de Oliveira)},
   Creator={pdfLaTeX},
   Producer={pdfTeX},
   CreationDate={},
   ModDate={},
   Subject={Estudo sobre algoritmos},
   Keywords={algoritmos, algorithms, homework, exerc\'{i}cios, respostas, solutions},
   }}
  %\usepackage{thumbpdf}
  \hypersetup{colorlinks,%
    debug=false,%
    linkcolor=blue,%
    citecolor=blue,%
    urlcolor=blue}
  \usepackage{cleveref}
  \mathchardef\period=\mathcode`.
\else
  \documentclass[a4paper,12pt]{article}
  \usepackage[utf8]{inputenc}
  \usepackage[T1]{fontenc}
  \usepackage[brazil]{babel}
  \usepackage[dvips]{geometry}
  \usepackage[dvips]{graphicx}
  \geometry{a4paper,left=2cm,right=2cm,top=2cm,bottom=2cm}
  \usepackage{setspace}
  \usepackage{varioref}
  \usepackage{hyperref}
  \usepackage{cleveref}
\fi


%%%%%%%%%%%%%%%%%%%%%%%%%%%%%%%%%%%%%%%%%%%%%%%
%%% Configura lingua portuguesa:
%\usepackage[brazil]{babel}
%\usepackage[utf8]{inputenc}
%\usepackage[T1]{fontenc}


%%%%%%%%%%%%%%%%%%%%%%%%%%%%%%%%%%%%%%%%%%%%%%%
%%% Altera fonte padrão
% phv=Helvetica ptm=Times ppl=Palatino pbk=bookman
% pag=AdobeAvantGarde pnc=Adobe NewCenturySchoolbook
\renewcommand{\familydefault}{ppl}


%%%%%%%%%%%%%%%%%%%%%%%%%%%%%%%%%%%%%%%%%%%%%%%
%%% Configura símbolos e bibliotecas matemáticas:
\usepackage{amsmath}
\usepackage{amssymb}
\usepackage{latexsym}
\usepackage{array}
\usepackage[ruled]{algorithm}
%\usepackage{physics}
\usepackage{siunitx}
\sisetup{group-separator = {.}}
\sisetup{group-digits = {false}}
\sisetup{output-decimal-marker = {,}}

\usepackage{listings}
\lstset{literate=
  {á}{{\'a}}1 {é}{{\'e}}1 {í}{{\'i}}1 {ó}{{\'o}}1 {ú}{{\'u}}1
  {Á}{{\'A}}1 {É}{{\'E}}1 {Í}{{\'I}}1 {Ó}{{\'O}}1 {Ú}{{\'U}}1
  {à}{{\`a}}1 {è}{{\`e}}1 {ì}{{\`i}}1 {ò}{{\`o}}1 {ù}{{\`u}}1
  {À}{{\`A}}1 {È}{{\'E}}1 {Ì}{{\`I}}1 {Ò}{{\`O}}1 {Ù}{{\`U}}1
  {ä}{{\"a}}1 {ë}{{\"e}}1 {ï}{{\"i}}1 {ö}{{\"o}}1 {ü}{{\"u}}1
  {Ä}{{\"A}}1 {Ë}{{\"E}}1 {Ï}{{\"I}}1 {Ö}{{\"O}}1 {Ü}{{\"U}}1
  {â}{{\^a}}1 {ê}{{\^e}}1 {î}{{\^i}}1 {ô}{{\^o}}1 {û}{{\^u}}1
  {Â}{{\^A}}1 {Ê}{{\^E}}1 {Î}{{\^I}}1 {Ô}{{\^O}}1 {Û}{{\^U}}1
  {œ}{{\oe}}1 {Œ}{{\OE}}1 {æ}{{\ae}}1 {Æ}{{\AE}}1 {ß}{{\ss}}1
  {ű}{{\H{u}}}1 {Ű}{{\H{U}}}1 {ő}{{\H{o}}}1 {Ő}{{\H{O}}}1
  {ç}{{\c c}}1 {Ç}{{\c C}}1 {ø}{{\o}}1 {å}{{\r a}}1 {Å}{{\r A}}1
  {€}{{\euro}}1 {£}{{\pounds}}1 {«}{{\guillemotleft}}1
  {»}{{\guillemotright}}1 {ñ}{{\~n}}1 {Ñ}{{\~N}}1 {¿}{{?`}}1
}


%%%%%%%%%%%%%%%%%%%%%%%%%%%%%%%%%%%%%%%%%%%%%%%
%%% Configura fontes e outros símbolos
\usepackage{wasysym}
\usepackage{pifont}
\usepackage{marvosym}


%%%%%%%%%%%%%%%%%%%%%%%%%%%%%%%%%%%%%%%%%%%%%%%
%%% Ativa pacote ifthen, necessário para alguns comandos
\usepackage{ifthen}


%%%%%%%%%%%%%%%%%%%%%%%%%%%%%%%%%%%%%%%%%%%%%%%
%%% Ativa suporte a cores:
\usepackage{color}
\usepackage[dvipsnames]{xcolor}
\usepackage{xparse}


%%%%%%%%%%%%%%%%%%%%%%%%%%%%%%%%%%%%%%%%%%%%%%%
%%% Ativa figuras e tabelas
\usepackage{float}
\usepackage{wrapfig}


%%%%%%%%%%%%%%%%%%%%%%%%%%%%%%%%%%%%%%%%%%%%%%%
%%% Ativa suporte ao TikZ Code
\usepackage{tikz}
\usetikzlibrary{positioning,shapes,shadows}


%%%%%%%%%%%%%%%%%%%%%%%%%%%%%%%%%%%%%%%%%%%%%%%
%%% Ativa pacote para tabelas longas e em landscape
\usepackage{array,longtable}
\usepackage{lscape}
\usepackage{array}
\usepackage{colortbl}
\newcolumntype{M}[1]{>{\centering\arraybackslash}m{#1}}
%\newcolumntype{ML}[1]{>{$}l<{$}}
%\newcolumntype{MR}[1]{>{R}r<{R}}
\newcolumntype{L}[1]{>{\arraybackslash}m{#1}}
\newcolumntype{N}{@{}m{0pt}@{}}


%%%%%%%%%%%%%%%%%%%%%%%%%%%%%%%%%%%%%%%%%%%%%%%
%%% Ativa pacote para URLs, e-mails e pathmanes:
\usepackage{url}


%%%%%%%%%%%%%%%%%%%%%%%%%%%%%%%%%%%%%%%%%%%%%%%
%%% Commando para ``italizar´´ palavras em inglês (e outras línguas!)
\newcommand{\ingles}[1]{\textit{#1}}


%%%%%%%%%%%%%%%%%%%%%%%%%%%%%%%%%%%%%%%%%%%%%%%
%%% Commando para colocar o espaço correto entre um número e sua unidade
\newcommand{\unidade}[2]{\ensuremath{#1\,\mathrm{#2}}}
\newcommand{\unidado}[2]{{#1}\,{#2}}


%%%%%%%%%%%%%%%%%%%%%%%%%%%%%%%%%%%%%%%%%%%%%%%%%%%%%%%%%%%%
%% produz ordinal masculino ou feminino dependendo do segundo
%% argumento.  Por exemplo:
%% \ordinal{1}{a} Semana
%% \ordinal{1}{o} Encontro
\newcommand{\ordinal}[2]{%
#1%
\ifthenelse{\equal{a}{#2}}%
{\textordfeminine}%
{\textordmasculine}}


%%%%%%%%%%%%%%%%%%%%%%%%%%%%%%%%%%%%%%%%%%%%%%%
%%% Ativa suporte a sublinhado:
% A opção normalem indica que ênfase será dada por itálico
% e não por sublinhado.
\usepackage[normalem]{ulem}
% O código a seguir define mais comandos para o pacote ulem, e foi retirado
% do "LaTeX demo: exemplos com LaTeXe", de Klauss Steding Jessen:
\def\dotuline{\bgroup
  \ifdim\ULdepth=\maxdimen  % Set depth based on font, if not set already
  \settodepth\ULdepth{(j}\advance\ULdepth.4pt\fi
  \markoverwith{\begingroup
  \advance\ULdepth0.08ex
  \lower\ULdepth\hbox{\kern.15em .\kern.1em}%
  \endgroup}\ULon}
\def\dashuline{\bgroup
  \ifdim\ULdepth=\maxdimen  % Set depth based on font, if not set already
  \settodepth\ULdepth{(j}\advance\ULdepth.4pt\fi
  \markoverwith{\kern.15em
  \vtop{\kern\ULdepth \hrule width .3em}%
  \kern.15em}\ULon}


%%%%%%%%%%%%%%%%%%%%%%%%%%%%%%%%%%%%%%%%%%%%%%%
%%% Ativa pacote para indentação da primeira linha de parágrafos
%\usepackage{indentfirst}


%%%%%%%%%%%%%%%%%%%%%%%%%%%%%%%%%%%%%%%%%%%%%%%
%%% Ativa pacote enumerate, extensão ao environment enumerate:
\usepackage{enumerate}


%%%%%%%%%%%%%%%%%%%%%%%%%%%%%%%%%%%%%%%%%%%%%%%
%%% environment ``Description'', similar ao environment
%%% ``description'', mas com maior controle sobre a tabulação das
%%% entradas e de suas descrições.
%%% Adaptado de um exemplo do LaTeX Companion, pg. 64.
\newlength{\myentrylen}
\newenvironment{Description}[1]%
{\list{}
  {\settowidth{\labelwidth}{\textbf{#1}}%
    \leftmargin\labelwidth\advance\leftmargin\labelsep%
    \renewcommand{\makelabel}[1]{%
      \settowidth{\myentrylen}{\textbf{##1}}%
      \ifthenelse{\lengthtest{\myentrylen > \labelwidth}}%
      {\parbox[b]{\labelwidth}%
        {\makebox[0pt][l]{\textbf{##1}}\\\mbox{}}}
      {\textbf{##1}}%
      \hfill\relax%
      }
}}
{\endlist}


%%%%%%%%%%%%%%%%%%%%%%%%%%%%%%%%%%%%%%%%%%%%%%%
%%% Comando para epígrafe em capítulos/seções (não confundir com a epígrafe geral
% da tese, definida em página isolada nos elementos pré-textuais.
\newcommand{\epigrafe}[2]{
   \vspace{-6ex}%
     {\footnotesize%
     \begin{flushright}%
     \begin{minipage}{.6\textwidth}%
     #1
     \end{minipage}\\
     \textit{#2}%
     \end{flushright}}%
   \vspace{-3ex}}


%%%%%%%%%%%%%%%%%%%%%%%%%%%%%%%%%%%%%%%%%%%%%%%
%%% Ativa pacote para controle de cabeçalhos e rodapés e configura;
% Ativa pacote:
\usepackage{fancyhdr}
% Configura estilo padrão das páginas
\pagestyle{headings}


%%%%%%%%%%%%%%%%%%%%%%%%%%%%%%%%%%%%%%%%%%%%%%%
%%% Ativa pacote para formatar os captions:
\usepackage[normal,bf]{caption}
%\captionsetup[table]{font=small,skip=0pt}
\captionsetup[figure]{skip=0pt}


%%%%%%%%%%%%%%%%%%%%%%%%%%%%%%%%%%%%%%%%%%%%%%%
%%% Ativa o MakeIndex para fazer índices remissivos:
\usepackage{makeidx}
%\makeindex


%%%%%%%%%%%%%%%%%%%%%%%%%%%%%%%%%%%%%%%%%%%%%%%
%%% Ativa pacote para fazer glossário, conforme
% "LaTeX demo: exemplos com LaTeXe", de Klauss Steding Jessen.
\usepackage{makeglo}
%\makeglossary


%%%%%%%%%%%%%%%%%%%%%%%%%%%%%%%%%%%%%%%%%%%%%%%
%%% Ativa o pacote havard de referências bibliográficas e
% define um novo comando para colocar as citações em slanted:
% As opções do pacote determinam como as citações aparecem no texto.
% As seguinte opções existem:
%      default: lista a primeira completa e as subsequentes abreviadas;
%	  full: lista todas as citações completas;
%         abbr: lista todas as citações abreviadas.
% A qualquer momento o modo de citaçõa pode ser alterado com o uso do
% comando: \citationmode{}, cujo argumento é uma das opções da lista anterior.
\usepackage[default]{harvard}
% Cria comando para colocar as citações em slanted:
\newcommand{\refbib}[1]{\textsl{#1}}
% O seguinte comando configura como as referências bibliográficas serão
% formatadas. O estilo agsm é o padrão do pacote havard. Ver manual de instrução
% do pacote para maiores informações.
\bibliographystyle{agsm}
% Configura como as citações das referências apareceram no texto. O estilo agsm
% é o padrão do pacote havard. Ver manual de instrução do pacote para maiores
% informações.
\citationstyle{agsm}


%%%%%%%%%%%%%%%%%%%%%%%%%%%%%%%%%%%%%%%%%%%%%%%
%%% Comandos específicos para este documento


%%%%%%%%%%%%%%%%%%%%%%%%%%%%%%%%%%%%%%%%%%%%%%%
%%% Determina forma de hifenização de palavras quando a hifenização
%%% padrão não estiver correta
%\hyphenation{ne-nhu-ma}
\babelhyphenation[brazil]{ne-nhu-ma Git-Hub}


%%%%%%%%%%%%%%%%%%%%%%%%%%%%%%%%%%%%%%%%%%%%%%%%%%%%%%%%%%%%%%%%%%%%%%%%%%%%%%%%%%%%%%%%%%%%%%
%%%%%%%%%%%%%%%%%%%%%%%%%%%%%%%%%%%%%%%%%%%%%%%%%%%%%%%%%%%%%%%%%%%%%%%%%%%%%%%%%%%%%%%%%%%%%%
%%%%%%%%%%%%%%%%%%%%%%%%%%%%%%%%%%%%%%%%%%%%%%%%%%%%%%%%%%%%%%%%%%%%%%%%%%%%%%%%%%%%%%%%%%%%%%
%%%%%%%%%%%%%%%%%%%%%%%%%%%%%%%%%%%%%%%%%%%%%%%%%%%%%%%%%%%%%%%%%%%%%%%%%%%%%%%%%%%%%%%%%%%%%%
%%%%%%%%%%%%%%%%%%%%%%%%%%%%%%%% COMEÇA DOCUMENTO %%%%%%%%%%%%%%%%%%%%%%%%%%%%%%%%%%%%%%%%%%%%
%%%%%%%%%%%%%%%%%%%%%%%%%%%%%%%%%%%%%%%%%%%%%%%%%%%%%%%%%%%%%%%%%%%%%%%%%%%%%%%%%%%%%%%%%%%%%%
%%%%%%%%%%%%%%%%%%%%%%%%%%%%%%%%%%%%%%%%%%%%%%%%%%%%%%%%%%%%%%%%%%%%%%%%%%%%%%%%%%%%%%%%%%%%%%
%%%%%%%%%%%%%%%%%%%%%%%%%%%%%%%%%%%%%%%%%%%%%%%%%%%%%%%%%%%%%%%%%%%%%%%%%%%%%%%%%%%%%%%%%%%%%%
%%%%%%%%%%%%%%%%%%%%%%%%%%%%%%%%%%%%%%%%%%%%%%%%%%%%%%%%%%%%%%%%%%%%%%%%%%%%%%%%%%%%%%%%%%%%%%
\begin{document}
\title{\emph{Conceitos de Algoritmos},\\
  Daniel Barbosa de Oliveira (2017)\\
  \ \\
--- respostas aos exercícios sobre expressões ---}
\author{Abrantes Araújo Silva Filho}
\date{2018-03}
\maketitle
\tableofcontents
%\newpage


%%%%%%%%%%%%%%%%%%%%%%%%%%%%%%%%%%%%%%%%%%%%%%%%%%%%%%%%%%%%%%%%%%%%%%%%%%%%%%%%%%%%%%%%%%%%%%
%%%%%%%%%%%%%%%%%%%%%%%%%%%%%%%%%%%%%%%%%%%%%%%%%%%%%%%%%%%%%%%%%%%%%%%%%%%%%%%%%%%%%%%%%%%%%%
%%%%%%%%%%%%%%%%%%%%%%%%%%%%%%%%%%%%%%%%%%%%%%%%%%%%%%%%%%%%%%%%%%%%%%%%%%%%%%%%%%%%%%%%%%%%%%
%%%%%%%%%%%%%%%%%%%%%%%%%%%%%%%%%%%%%%%%%%%%%%%%%%%%%%%%%%%%%%%%%%%%%%%%%%%%%%%%%%%%%%%%%%%%%%
%%%%%%%%%%%%%%%%%%%%%%%%%%%%%%%%%%%%%%%%%%%%%%%%%%%%%%%%%%%%%%%%%%%%%%%%%%%%%%%%%%%%%%%%%%%%%%
\section{O que é este documento?} 
\label{o_que_e}
%\thispagestyle{plain}

Este documento contém as minhas respostas aos exercícios e problemas da seção ``Expressões''
(página 31) da apostila \emph{Conceitos de Algoritmos}, de Daniel Barbosa de OLiveira (2017),
utilizada na disciplina de Algoritmos-I (gradução em Ciência da Computação, Faesa).

ATENÇÃO: não garanto que tudo aqui está correto, pelo contrário, algumas respostas expressam
minha visão particular e podem estar em desacordo com
a ``resposta padrão'' dos autores do livro ou do professor da disciplina de Algoritmos. Também
não garanto que todos os exercícios e problemas do capítulo ou livro estarão resolvidos aqui.
De qualquer modo, caso pretenda
utilizar este documento como base para seu próprio estudo, tenha em mente o seguinte:

\begin{quote}
  \emph{Este documento é fornecido ``no estado em que se encontra'', sem garantias de qualquer
    natureza, expressas ou implícitas. Em nenhuma hipótese o autor poderá ser responsabilizado
    por qualquer problema, dano, prejuízo material ou imaterial decorrente do uso deste
    conteúdo.}
\end{quote}

Este documento (em formato PDF), o original em \LaTeX\ e outros materiais
adicionais (se necessário) estão disponíveis no seguinte
repositório GitHub: \url{https://github.com/abrantesasf/algoritmos}


%%%%%%%%%%%%%%%%%%%%%%%%%%%%%%%%%%%%%%%%%%%%%%%%%%%%%%%%%%%%%%%%%%%%%%%%%%%%%%%%%%%%%%%%%%%%%%
%%%%%%%%%%%%%%%%%%%%%%%%%%%%%%%%%%%%%%%%%%%%%%%%%%%%%%%%%%%%%%%%%%%%%%%%%%%%%%%%%%%%%%%%%%%%%%
%%%%%%%%%%%%%%%%%%%%%%%%%%%%%%%%%%%%%%%%%%%%%%%%%%%%%%%%%%%%%%%%%%%%%%%%%%%%%%%%%%%%%%%%%%%%%%
%%%%%%%%%%%%%%%%%%%%%%%%%%%%%%%%%%%%%%%%%%%%%%%%%%%%%%%%%%%%%%%%%%%%%%%%%%%%%%%%%%%%%%%%%%%%%%
%%%%%%%%%%%%%%%%%%%%%%%%%%%%%%%%%%%%%%%%%%%%%%%%%%%%%%%%%%%%%%%%%%%%%%%%%%%%%%%%%%%%%%%%%%%%%%
\section{Exercícios} 
\label{exercicios}
%\thispagestyle{plain}


%%%%%%%%%%%%%%%%%%%%%%%%%%%%%%%%%%%%%%%%%%%%%%%%%%%%%%%%%%%%%%%%%%%%%%%%%%%%%%%%%%%%%%%%%%%%%%
%%%%%%%%%%%%%%%%%%%%%%%%%%%%%%%%%%%%%%%%%%%%%%%%%%%%%%%%%%%%%%%%%%%%%%%%%%%%%%%%%%%%%%%%%%%%%%
\subsection{Exercício 1: escreva as expressões na forma computacional}
\label{grupo_1}

\paragraph{a)}

$$ae + \frac{b^{(3x)}}{c^2} = \verb|a*e + (b^(3*x)/c^2)|$$

\paragraph{b)} $$\frac{2x^2 - (3x)^{(x + 1)}}{2} + \frac{sqrt{x + 1}}{x^2} =
\verb!((2*x^2 - (3*x)^(x + 1))/2) + ((sqrt(x + 1))/x^2)!$$

\paragraph{c)} $$2h - \left(\frac{45}{3x}-4h(3-h)\right)^{22k} = \verb!2*h - ((45/(3*x)) - 4*h*(3 - h))^(22*k)!$$

\paragraph{d)} $$\frac{\sqrt{2b-4a^2} + 2f^{-3}}{3-2a} = \verb!(sqrt(2*b - 4*a^2) + 2*f^(-3))/(3 - 2*a)!$$

\paragraph{e)} $$ \frac{\sqrt{-6^x + (2y)^{\frac{1}{3}}}}{3^9} = \verb!(sqrt(-6^x + (2*y)^(1/3)))/3^9!$$

\paragraph{f)} $$\sqrt{\frac{2x + u^{\frac{2}{3}}}{a + bc}} = \verb!sqrt((2*x + u^(2/3))/(a+b*c))!$$


%%%%%%%%%%%%%%%%%%%%%%%%%%%%%%%%%%%%%%%%%%%%%%%%%%%%%%%%%%%%%%%%%%%%%%%%%%%%%%%%%%%%%%%%%%%%%%
%%%%%%%%%%%%%%%%%%%%%%%%%%%%%%%%%%%%%%%%%%%%%%%%%%%%%%%%%%%%%%%%%%%%%%%%%%%%%%%%%%%%%%%%%%%%%%
\subsection{Exercício 2: escreva as expressões em formato tradicional}
\label{grupo_2}

\paragraph{a)} $$\verb!a+b+(34+exp(e,9))/u-89^(1/2)! = a + b + \frac{34+e^9}{u} - 89^{\frac{1}{2}}$$

\paragraph{b)} $$\verb!23+5/((7*a)/47)^(2/x)! = 23 + \frac{5}{\left(\frac{7a}{47}\right)^{\frac{2}{x}}} $$

\paragraph{c)} $$\verb!(12*x)/(36-9^x)^2! = \frac{12x}{(36-9^x)^2}$$


%%%%%%%%%%%%%%%%%%%%%%%%%%%%%%%%%%%%%%%%%%%%%%%%%%%%%%%%%%%%%%%%%%%%%%%%%%%%%%%%%%%%%%%%%%%%%%
%%%%%%%%%%%%%%%%%%%%%%%%%%%%%%%%%%%%%%%%%%%%%%%%%%%%%%%%%%%%%%%%%%%%%%%%%%%%%%%%%%%%%%%%%%%%%%
\subsection{Exercício 3: resolva as expressões lógicas}
\label{grupo_3}

Atenção: nos exercícios a seguir a apostila é ambígua em relação aos operadores ``\verb!\!" e
``\%''. Considerei portanto o seguinte:

\begin{itemize}
\item \verb!\! corresponde ao operador divisão inteira, definido como $a\verb!\!b \equiv \left \lfloor{a/b}\right \rfloor$
  (sendo $\lfloor{x}\rfloor$ a função floor).
  \item \% corresponde ao resto da divisão.
\end{itemize}

\paragraph{a)}
\begin{equation*}
  \begin{split}
    \text{Não } (2^3 < \sqrt{16} & \text{ ou } \left \lfloor{15/2}\right \rfloor < 10)\\
    \text{Não } (8 < 4 & \text{ ou } 7 < 10)\\
    \text{Não } (F     & \text{ ou } V)\\
    \text{Não } & (V) = F
  \end{split}
\end{equation*}

\paragraph{b)}
\begin{equation*}
  \begin{split}
    (6 < 8) & \text{ ou } (3 > 7)\\
    (V) & \text{ ou } (F) = V
  \end{split}
\end{equation*}

\paragraph{c)}
\begin{equation*}
  \begin{split}
    \text{Não } & (2 < 3)\\
    \text{Não } & (V) = F
  \end{split}
\end{equation*}

\paragraph{d)} Sabendo-se que A = 5:
\begin{equation*}
  \begin{split}
    (5 >= 6) \text{ ou } (6 < 7) & \text{ ou } \text{ não } (a + 5 - 6 = 8)\\
    (F) \text{ ou } (V)          & \text{ ou } \text{ não } (4 = 8)\\
    (V)                          & \text{ ou } \text{ não } (F)\\
    (V)                          & \text{ ou } (V) = V
  \end{split}
\end{equation*}

\paragraph{e)} Sabendo-se que U = 29:
\begin{equation*}
  \begin{split}
    (34 > 9 \text{ e } 5 + u = 34) & \text{ ou } (5 = 15/3 \text{ e } 8 > 12)\\
    (34 > 9 \text{ e } 34 = 34)    & \text{ ou } (V \text{ e } F)\\
    (V \text{ e } V)               & \text{ ou } (F)\\
    (V)                          & \text{ ou } (F) = V
  \end{split}
\end{equation*}

\paragraph{f)}
\begin{equation*}
  \begin{split}
    10\%4 & < 16\%3\\
    2 & < 1 = F
  \end{split}
\end{equation*}

\paragraph{g)}
\begin{equation*}
  \begin{split}
    2 + 8 \% 7 & >= 6 - \left (8^{(2/3)}\right)^{(1/2)}\\
    2 + 1      & >= 6 - \left (\sqrt[3]{8^2}\right)^{(1/2)}\\
    3          & >= 6 - \sqrt{\sqrt[3]{64}}\\
    3          & >= 6 - \sqrt{4}\\
    3          & >= 6 - 2\\
    (3          & >= 4) = F
  \end{split}
\end{equation*}

\paragraph{h)}
\begin{equation*}
  \begin{split}
    \left \lfloor{15/7}\right \rfloor & >= 27 \% 5\\
    (2 & >= 2 )= V
  \end{split}
\end{equation*}



%%%%%%%%%%%%%%%%%%%%%%%%%%%%%%%%%%%%%%%%%%%%%%%%%%%%%%%%%%%%%%%%%%%%%%%%%%%%%%%%%%%%%%%%%%%%%%
%%%%%%%%%%%%%%%%%%%%%%%%%%%%%%%%%%%%%%%%%%%%%%%%%%%%%%%%%%%%%%%%%%%%%%%%%%%%%%%%%%%%%%%%%%%%%%
%%%%%%%%%%%%%%%%%%%%%%%%%%%%%%%%%%%%%%%%%%%%%%%%%%%%%%%%%%%%%%%%%%%%%%%%%%%%%%%%%%%%%%%%%%%%%%
%%%%%%%%%%%%%%%%%%%%%%%%%%%%%%%%%%%%%%%%%%%%%%%%%%%%%%%%%%%%%%%%%%%%%%%%%%%%%%%%%%%%%%%%%%%%%%
%%%%%%%%%%%%%%%%%%%%%%%%%%%%%%%%%%%%%%%%%%%%%%%%%%%%%%%%%%%%%%%%%%%%%%%%%%%%%%%%%%%%%%%%%%%%%%
%\newpage
%\section{Problemas}
%\label{problemas}
%\thispagestyle{plain}


%%%%%%%%%%%%%%%%%%%%%%%%%%%%%%%%%%%%%%%%%%%%%%%%%%%%%%%%%%%%%%%%%%%%%%%%%%%%%%%%%%%%%%%%%%%%%%
%\subsection{Problema 2.1}
%\label{problema_2_1}


%%%%%%%%%%%%%%%%%%%%%%%%%%%%%%%%%%%%%%%%%%%%%%%
%%% Produz as referências bibliográficas
%% Configura título das referências bibliográficas:
%\newpage
%\renewcommand{\refname}{Referências bibliográficas}
%% Ativa arquivo com as referências bibliográficas:
%\bibliography{evandro}
%% Adiciona entrada na toc:
%\addcontentsline{toc}{section}{Referências bibliográficas}
%% Estilo da página
%\thispagestyle{headings}
%\index{referencias bibliograficas@Referências bibliográficas}


% Termina o documento
\end{document}             
